\chapter{Wykład 6. Systematyczny opis metodyki SCRUM w zarządzaniu projektami}

\section{Czy warto wprowadzić metodykę SCRUM?}
% strona 89

Scrum jest lekką metodyką zarządzania projektami.  Metoda ta promuje pracę iteracyjną, czyli podzieloną na 2-6 tygodniowe okresy pracy zwane sprintami. Projekt nie jest planowany z góry na cały okres trwania, lecz przed każdym sprintem następuje planowanie na najbliższe tygodnie. Dzięki temu Scrum jest metodyką otwartą na zmiany, po każdym sprincie mogą zmieniać się wymagania. 

Nasz projekt tworzony jest po raz pierwszy, nie ma możliwości odwołania się do podobnych produktów wytworzonych w przeszłości. Z tego powodu, w naszej sytuacji dużą zaletą Scrum’a jest częsty kontakt z klientem, który otrzymuje pewne funkcjonalności produktu po każdym sprincie.  Może je ocenić i zweryfikować czy są zgodne z jego oczekiwaniami. Nam pozwoli to wytworzyć produkt w pełni spełniający wymagania użytkownika. Dodatkowo Scrum jest metodyką, w której nacisk kładziony jest na komunikacje wewnątrz zespołu. Pomocne są codzienne Scrum Meetingi, na których członkowie zespołu mówią o postępach prac i napotkanych problemach. Naszym zdaniem, scaliłoby to dodatkowo młody zespół. 

Naszym zdaniem warto wprowadzić metodykę Scrum w naszym projekcie. Warunki jakie musiałyby być spełnione to: zatrudnienie lub przekwalifikowanie któregoś z pracowników na stanowisko Scrum Mastera,  wybór Scrum Product Ownera, sprinty trwające 3 tygodnie, ustalenie sztywnych godzin pracy w celu realizacji porannych scrum meetingów, planowanie przed każdym sprintem metodą pokerową.

\clearpage

\section{Inicjacja projektu w SCRUM}
% strona 95

W inicjacji projektu (opracowaniu jego wizji) w metodyce SCRUM uczestniczą: właściciel biznesu (finansujący projekt), właściciel produktu, interesariusze oraz zespół projektowy (celem estymacji wykonalności i czasochłonności). W naszym przypadku, w spotkaniu uczestniczyłyby następujące osoby: klient, zarząd naszej firmy, zespół projektowy (analitycy, projektanci, programiści, testerzy).

Podczas inicjacji należy:
\begin{itemize}
\item ustalić kim będą użytkownicy,
\item ustalić jakie potrzeby klienta ma spełniać produkt,
\item przeanalizować jakie są produkty konkurencji, jaka będzie szacowana cena i jakie są możliwości zdobywania rynku,
\item przygotować budżet rozwoju produktu (opłacalność rozwoju produktu),
\item przygotować plan rozwoju produktu (liczba, cele, czas trwania kolejnych Sprintów),
\item przygotować wstępny rejestr produktowy dla projektu (lista cech, oszacowanie czasu i kolejności realizacji).
\end{itemize}

