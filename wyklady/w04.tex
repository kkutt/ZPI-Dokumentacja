
\chapter{Wykład 4. Zarządzanie wiedzą w przedsiębiorstwach}

\section{Procesy wykorzystania produktu projektu}
% strona 34

Zobacz Rys.~\ref{fig:opracowanieWBS}~i~\ref{fig:zarzadzanieRyzykiem}.

\begin{figure}[hbt]
\centering
\includegraphics[width=1.1\textwidth]{opracowywanieWBS.png}
\caption{Tworzenie WBS}
\label{fig:opracowanieWBS}
\end{figure}

\begin{figure}[hbt]
\centering
\includegraphics[width=1.1\textwidth]{zarzadzanieRyzykiem.png}
\caption{Zarządzanie ryzykiem}
\label{fig:zarzadzanieRyzykiem}
\end{figure}

% ===========================================================================

\section{Model przepływu danych}
% strona 40

Ten wirtualny warsztat jest beznadziejny.

% ===========================================================================

\section{Mapa umysłu dla systemu zarządzania wiedzą}
% strona 70

Zobacz Rys.~\ref{fig:mapaMysli}

\begin{figure}[hbt]
\centering
\includegraphics[width=\textwidth]{mapaMysli.png}
\caption{Mapa myśli}
\label{fig:mapaMysli}
\end{figure}

% ===========================================================================

\section{Przegląd praktyk OPM3}
% strona 89



Praktyki OPM3 które powinny być w firmie:

\begin{enumerate}

\item Integrate PMBOK Guide Knowledge Areas; z racji związania projektu z metodyką PMBOK

\item Project Team Development Process Measurement; w związku z pracą zespołową nad projektem

\item Project Risk Response Planning Process Control; związane z występowaniem ryzyka

\end{enumerate}

Praktyki OPM3 zbędne w firmie:

\begin{enumerate}

\item Know Inter-Project Plan; w trakcie trwania projektu nie będą prowadzone równolegle inne projekty

\item Optimize Portfolio Management; brak portfolio

\item Track the Return of Investment; projekt nie jest inwestycją firmy

\end{enumerate}