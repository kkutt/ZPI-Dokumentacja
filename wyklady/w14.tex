\chapter{Wykład 14. Zarządzanie ryzykiem w projekcie informatycznym}

\section{Macierz ryzyka}
% strona 27

Zidentyfikowane ryzyka:
\begin{enumerate} %nie zmieniać kolejności, bo te liczby są w tabeli!
\item Nieznajomość wybranego w projekcie języka programowania
\item Niedyspozycja członka zespołu (choroba, awaria sprzętu)
\item Awaria repozytorium (svn)
\item Skrócenie czasu realizacji projektu
\item Klęska żywiołowa
\item Problemy merytoryczne związane z brakiem doświadczenia w tematyce projektu
\item Niespełnienie wymagań jakościowych (błędne działanie na różnych systemach operacyjnych, wolne działanie aplikacji, zawodność)
\item Zbyt długi czas realizacji projektu
\item Skończenie się środków z odszkodowania które zostały przeznaczone na projekt
\item Niemożliwość znalezienia kupca chętnego na projekt po jego zakończeniu
\item Odejście przed zakończeniem projektu, któregoś z członków grupy projektowej
\item Otrzymanie innego dużego zlecenia w trakcie prac nad dokończeniem projektu
\item Przedłużający się proces z inwestorem w sprawie odszkodowania
\item Ukazanie się we wcześniejszym terminie produktu konkurencyjnej firmy
\item Wewnętrzna sytuacja w firmie zmuszająca do przeznaczenia środków z odszkodowania na inne cele
\item Pomimo znalezienia klientów inwestycja nie zwraca się
\end{enumerate}


\begin{table}[htb]
\centering
\begin{tabular}{|c|c|c|c|c|c|c|} 
\multicolumn{3}{|c|}{Szanse} &  & \multicolumn{3}{|c|}{Zagrożenia} \\
\hline  &  &  & Mocny & 1,10,11 & 3,7,9 &  \\
\hline  & 4 &  & Umiarkowany & 5,14,15 & 2,6,8 & 13 \\
\hline  &  &  & Słaby &  & 12,16 &  \\
\hline Wysokie & Średnie & Niskie &  & Niskie & Średnie & Wysokie \\
\end{tabular}
\caption{\textbf{Macierz prawdopodobieństwa dyskretna (trój-poziomowa)}. \mbox{\textit{Prawodopodobieństwo:}} oś pozioma,  \textit{Wpływ:} oś pionowa}
\label{tab:macierzPrawdopodobienstwa}
\end{table}


\begin{table}[htb]
\centering
\begin{tabular}{|c|c|c|c|} 
\hline Mocny & III & IV & V \\
\hline Umiarkowany & II & III & IV \\
\hline Słaby & I & II & III \\
\hline & Niskie & Średnie & Wysokie \\
\end{tabular}
\caption{\textbf{Poziomy ryzyka}}
\label{tab:poziomyRyzyka}
\end{table}


Definicje poziomów ryzyka:
\begin{enumerate}[I]
\item \textbf{Ryzyko tolerowane}. Posiada mały lub żaden efekt na cele projektu. Prawdopodobieństwo jest na tyle małe, że nie ma potrzeby ich rozpatrywania.
\item \textbf{Ryzyko małe}. Mały wpływ na cele projektu. Prawdopodobieństwo wystąpienia małe w
związku z tym niezbyt duża koncentracja nad ryzykiem.
\item \textbf{Ryzyko średnie}. Może wpłynąć na cele projektowe, harmonogram i koszty. Prawdopodobieństwo pojawienia się jest na tyle wysokie, że musi podlegać ścisłej kontroli wszystkich czynników wpływających na ryzyko.
\item \textbf{Ryzyko wysokie}. Wysokie prawdopodobieństwo pojawienia się, wpływające na cele projektu, koszt i harmonogram. Ryzyko musi podlegać ścisłej kontroli określenia akcji zapobiegawczych.
\item \textbf{Ryzyko nietolerowane}. To ryzyko należy do najważniejszych zagrożeń i szans w projekcie.
\end{enumerate}

\clearpage

% ===========================================================================

\section{Analiza jakościowa i ilościowa SWOT}
% strona 48

\begin{figure}[!h]
\begin{center}
\includegraphics[scale=1]{swot.png}
\caption[Analiza SWOT]{Analiza SWOT}
\label{rysunekProces}
\end{center}
\end{figure}

\clearpage

% ===========================================================================

\section{Analiza jakościowa ryzyka}
% strona 54

Na pół roku przed końcem projektu, inwestor wycofuje się z projektu, zarzucając go.
\begin{itemize}
\item zgodnie z zapisem w umowie otrzymujemy 200 000 tys zł odszkodowania, za zerwanie umowy
\item postanawiamy dokończyć projekt w przeciągu pół roku, a następnie znaleźć chętnych na niego klientów
\item PMBOK jest popularny, więc powinniśmy znaleźć przedsiębiorstwo które pracuje zgodnie z nim
\item Zarzucenie projektu, który jest w wysokim stopniu zaawansowania prac jest nieopłacalne
\end{itemize}


Prawdopodobieństwo:  M - małe, S – średnie, D – duże\\
Wpływ: N - niski, P – poważny, W – wielki

\begin{table}[htb]
\centering
\begin{tabular}{|c|p{11.5cm}|c|c|} 
\hline Lp. & Opis & Prawdop. & Wpływ \\
\hline 1 & Zbyt długi czas realizacji projektu & S & P \\
\hline 2 & Skończenie się środków z odszkodowania które zostały przeznaczone na projekt & S & W \\
\hline 3 & Niemożliwość znalezienia kupca chętnego na projekt po jego zakończeniu & M & W \\
\hline 4 & Odejście przed zakończeniem projektu, któregoś z członków grupy projektowej & M & W \\
\hline5 & Długotrwała choroba kogoś z grupy projektowej & S & P \\
\hline6 & Otrzymanie innego dużego zlecenia w trakcie prac nad dokończeniem projektu & S & N \\
\hline7 & Przedłużający się proces z inwestorem w sprawie odszkodowania & D & P \\
\hline8 & Ukazanie się we wcześniejszym terminie produktu konkurencyjnej firmy & M & P \\
\hline9 & Wewnętrzna sytuacja w firmie zmuszająca do przeznaczenia środków z odszkodowania na inne cele  & M & P  \\
\hline10 & Pomimo znalezienia klientów inwestycja nie zwraca się &  S & N  \\
\hline
\end{tabular}
\caption{\textbf{Analiza jakościowa ryzyka}}
\label{tab:analizaJakosciowa}
\end{table}

\clearpage

\begin{table}[!h]
\centering
\begin{tabular}{|c|c|c|c|} \hline Prawdop. \textbackslash Wpływ & Niski & Poważny & Wielki \\
 \hline Duże &  & 7 &  \\
\hline Średnie & 6,10 & 1,5 & 2 \\
\hline Małe &  & 8,9  & 3,4  \\
\hline
\end{tabular}
\caption{\textbf{Macierz prawdopodobieństwa}}
\label{tab:macierzPrawdopodobienstwa}
\end{table}




% ===========================================================================

\section{Analiza ilościowa ryzyka}
% strona 62

\begin{table}[htb]
\centering
\begin{tabular}{|c|p{7.0cm}|c|c|c|c|}
\hline 
Lp. & Opis & Prawdop. & Kwota & Wart. efektu finans. \\
 \hline 1 & Zbyt długi czas realizacji projektu & 25\% & 200 000 zł  & 50 000 zł \\
\hline 2 & Skończenie się środków z odszkodowania które zostały przeznaczone na projekt & 10\% & 150 000 zł & 15 000 zł \\
\hline 3 & Długotrwała choroba kogoś z grupy projektowej &  30\%  & 30 000 zł & 10 000 zł \\
\hline 4 & Przedłużający się proces z inwestorem w sprawie odszkodowania & 50\% & 50 000 zł & 25 000 zł \\
\hline 5 & Wewnętrzna sytuacja w firmie zmuszająca do przeznaczenia środków z odszkodowania na inne cele  & 5\% & 150 000 zł & 7 500 zł  \\
\hline6 & Pomimo znalezienia klientów inwestycja nie zwraca się &  30\% & 100 000 zł & 30 000 zł  \\
\hline
\end{tabular}
\caption{\textbf{Analiza ilościowa ryzyka}}
\label{tab:analizaIlosciowa}
\end{table}

\clearpage

% ===========================================================================

\section{Plany reakcji na ryzyko}
% strona 75

\begin{figure}[h]
\begin{center}
\includegraphics[width=\textwidth]{planreakcji.png}
\caption[Plan reakcji na ryzyko]{Plan reakcji na ryzyko}
\label{rysunekProces}
\end{center}
\end{figure}


