\chapter{Wykład 11. Zarządzanie jakością w projekcie informatycznym}

\section{Lista kontrolna}
% strona 38

\textbf{Obszar projektu:} ocena wiarygodności estymacji harmonogramu i kosztu projektu

\begin{figure}[h]
\begin{center}
\includegraphics[scale=1]{checklist.png}
\caption[Lista kontrolna]{Lista kontrolna}
\label{rysunekProces}
\end{center}
\end{figure}

% ===========================================================================

\section{Plan poprawy procesów}
% strona 39

\textbf{Obszar projektu:} komunikacja

\textbf{Problem:} zespół nie może się dogadać, wymagania dotyczące projektu są błędnie interpretowane przez różne osoby, błędy jednej osoby pociągają za sobą błędy kolejnych, błędny przepływ informacji.

\textbf{Cel:} zgrany zespół, wymieniający się informacjami i problemami posiadający jasno określony cel działania znany wszystkim członkom zespołu

\begin{figure}[!h]
\begin{center}
\includegraphics[scale=1]{planpoprawy.png}
\caption[Plan poprawy procesu]{Plan poprawy procesu}
\label{rysunekProces}
\end{center}
\end{figure}
 


% ===========================================================================

\section{Plan zarządzania jakością pod kątem przydziału zasobów}
% strona 56

\begin{enumerate}
\item	Przygotowanie listy wszystkich zasobów na podstawie RBS.
\item	Bieżąca ocena zasobów ludzkich pod kątem kwalifikacji i stanowiska.  Sprawdzenie czy pracownicy wypełniają swoje obowiązki zawarte w opisie stanowiska oraz czy ich kwalifikacje pozwalają na wykonywanie danych czynności.
\item	Kontrola czy zasoby nie są przeciążone – czy nie jest im przypisane zbyt dużo pracy do wykonania, czy nie są zbyt eksploatowane.
\end{enumerate}

% ===========================================================================

\section{Audyt jakości}
% strona 62

Algorytm audytu jakości:
\begin{enumerate}

\item Tworzenie wzorca:

\begin{enumerate}

\item Opracowanie planu audytu.
\item Zlecenie wykonania audytu i ustalenie terminu badania.
\item Zabranie i analiza dokumentacji.
\item Przygotowanie listy pytań.

\end{enumerate}

\item Pomiar stanu faktycznego i porównanie ze wzorcem:

\begin{enumerate}

\item Spotkanie otwierające.
\item Wywiady, badania zapisów.

\end{enumerate}

\item Sprawdzenie odchyleń:

\begin{enumerate}

\item Określenie niezgodności.
\item Spotkanie zamykające.

\end{enumerate}


\item Klasyfikacja czynników zakłócających:

\begin{enumerate}

\item Opracowanie raportu i udokumentowanie niezgodności.
\item Działania korygujące i zapobiegawcze.

\end{enumerate}

\item Doprowadzenie systemu do stanu pożądanego:

\begin{enumerate}

\item Zatwierdzenie przeprowadzonego audytu.
\item Analiza procesu ciągłego doskonalenia.
\item Przegląd dokonywany przez kierownictwo.
\end{enumerate}
\end{enumerate}

Zasady audytowania:
\begin{itemize}
\item postępowanie etyczne:  zaufanie, rzetelność, poufność i rozwaga są integralną częścią audytowania;
\item rzetelna prezentacja: raporty i wszelkie ustalenia z audytu odzwierciedlają stan rzeczywisty, zgodnie z prawdą. Nierozstrzygnięte lub rozbieżne opinie pomiędzy zespołem audytującym a audytowanym są odnotowane w raporcie;
\item należyta staranność zawodowa: audytorzy wykazują staranność przy każdym zadaniu, niezależnie od poziomu zaawansowania, zgodnie ze swoimi kwalifikacjami kompetencjami zawodowymi;
\item niezależność:  audytorzy są niezależni od działalności poddanej audytowi są również wolni od uprzedzeń i konfliktu interesów. Zachowują obiektywizm podczas całego procesu audytu, aby zapewnić, że ustalenia i wnioski będą oparte tylko i wyłącznie na dowodach z audytu;
\item podejście oparte na dowodach:  dowód z audytu jest możliwy do zweryfikowania. Ponieważ audyt prowadzony jest w ograniczonym czasie z użyciem ograniczonych zasobów, oparty jest on na próbkach dostępnych informacji. Odpowiedni dobór próby związany jest ściśle z zaufaniem, jakie może mieć do wniosków z audytu.
\end{itemize}

\clearpage

% ===========================================================================

\section{Wyniki procesu kontroli jakości}
% strona 68
Dzięki audytowi możliwa jest kontrola procesów mających miejsce w przedsiębiorstwie. Audyty pozwalają na wykrycie niedoskonałości i błędów w działaniu. Każdy naprawiony defekt i dostarczony produkt również musi przejść przez kontrolę jakości. 

Dokument wyniku procesu kontroli mógłby wyglądać w następujący sposób:

\begin{figure}[!h]
\begin{center}
\includegraphics[width=\textwidth]{wyniki.png}
\caption[Szablon dokumentu]{Szablon dokumentu}
\label{rysunekProces}
\end{center}
\end{figure}

\clearpage

% ===========================================================================

\section{Diagram przyczynowo-skutkowy w zarządzaniu jakością}
% strona 80

Diagram przyczynowo-skutkowy jest jednym z narzędzi doskonalenia jakości. Pozwala na zidentyfikowanie przyczyny problemu i ułatwia znaleznienie przyczyny źródłowej problemu (root cause). Etapy tworzenia:
\begin{enumerate}
\item	Identyfikacja problemu (szary prostokąt).
\item	Określenie głównych grup przyczyny (niebieskie prostokąty)
\item	Uszczegółowienie przyczyn 
\item	Analiza wyników.
\end{enumerate}

\begin{figure}[!h]
\begin{center}
\includegraphics[width=\textwidth]{ryba.png}
\caption[Diagram przyczynowo-skutkowy]{Diagram przyczynowo-skutkowy}
\label{rysunekProces}
\end{center}
\end{figure}

\clearpage



% ===========================================================================

\section{Diagram Pareto}
% strona 83

\begin{figure}[!h]
\begin{center}
\includegraphics[width=\textwidth]{pareto.png}
\caption[Diagram Pareto]{Diagram Pareto}
\label{rysunekProces}
\end{center}
\end{figure}


Czy zasada 20-80 się sprawdza?

W wykonanym przykładzie zasada 20-80 nie sprawdziła się.  Około 80\% problemów było generowanych przez ok 44\% przyczyn.



